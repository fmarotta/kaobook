\renewcommand*{\chapterformat}
{
  \chapappifchapterprefix{\nobreakspace}\scalebox{3.5}{\thechapter\autodot}
}
\renewcommand\chapterlinesformat[3]
{
  %\vspace*{-1cm}%
  \leavevmode
  \makebox[\textwidth+\marginparsep+\marginparwidth]{%
	\makebox[\textwidth][l]{\hrulefill[1pt]#2}%\hfill%\par%\bigskip
	\makebox[\marginparsep][l]{}%
	\makebox[\marginparwidth][l]{}%
  }\\
  %\vspace{.5cm}
  %\makebox[\textwidth+\marginparsep+\marginparwidth]{%
	%\hrulefill[1pt]%
  %}\par
  \makebox[\textwidth+\marginparsep+\marginparwidth]{%
	\makebox[\textwidth][l]{#3}%
	\makebox[\marginparsep][l]{}%
	\makebox[\marginparwidth][l]{}%
  }\\
  \makebox[\textwidth+\marginparsep+\marginparwidth]{%
	\hrulefill[1.1pt]%
	%\makebox[\textwidth][l]{\hrulefill}%
	%\makebox[\marginparsep][l]{\hrulefill}%
	%\makebox[\marginparwidth][l]{\hrulefill}%
  }
}
\RedeclareSectionCommand[beforeskip=0cm]{chapter}

\setchapterpreamble[u]{\margintoc[*-5.5]}
\chapter{References}

\section{Citations}

To cite someone \sidecite{Visscher2008,James2013} is very simple: just 
use the \verb|\sidecite| command. It does not have an offset argument 
yet, but it probably will in the future. This command supports multiple 
entries, as you can see, and by default it prints the reference on the 
margin as well as adding it to the bibliography at the end of the 
document. For this setup I used biblatex but I think that workarounds 
are possible \sidecite{James2013}. Note that the citations have nothing 
to do with the text, they are completely random as they only serve the 
purpose to illustrate the feature.

\section{Glossaries and Indices}

If you load the packages \verb|glossaries| and \verb|imakeidx| you can 
add those things to your book. For instance, I previously defined some 
glossary entries and now I am going to use them, like this: 
\gls{computer}. \verb|glossaries| allows you to use acronyms as well, 
like the following: this is the full version, \acrfull{fpsLabel}, and 
this is the short one \acrshort{fpsLabel}. These entries will appear in 
the glossary in the backmatter.

To illustrate the indexing feature,\index{index} I have just called 
\verb|\index{index}|, and an entry in the index has been added. Check it 
out!

You can read the documentation of these packages if you are interested.

\setchapterpreamble[u]{\margintoc}
\chapter{Class Options}
\labch{options}

In this chapter I will describe the most common options used, both the 
ones inherited from scrbook and the kao-specific ones.

\section{KOMA options}

The class is based on the scrbook, therefore it understands all of the 
options you would normally pass to that class. By default, the font size 
is 9.5pt and the paragraphs are separated by space, not marked by 
indentation. The default value for parskip is half.

The toc has an entry for everything: listoffigures, listoftables, 
indices, glossaries and bibliographies. There are also entries for the 
table of contents itself (thanks to the 
\Command{setuptoc\{toc\}\{totoc\}} command). If you want entries for the 
glossaries as well, you can set the \Option{toc} option of the package 
\Package{glossaries}.\sidenote[-7mm][]{If you don't want all these 
things in the table of contents, pass the appropriate KOMA options to 
the class.}

\section{KAO options}

In the future I plan to add more options to set the paragraph formatting 
(\eg justified vs ragged) and the position of the margins (inner or 
outer in twoside mode, left or right in oneside 
mode)\sidenote[-10mm][]{As of now, paragraphs are justified, formatted 
with \Command{singlespacing} (from the \Package{setspace} package) and 
\Command{frenchspacing}.}. 

\section{Other things worth knowing}

By default, dispositions are numbered up to the section thanks to the 
command \Command{setcounter\{secnumdepth\}\{1\}}. The table of contents 
can be modified through the package \Package{etoc}, which is loaded 
because it is needed for the margintocs, or the more traditional 
\Package{tocbase}. The sidenotes are numbered on a per-chapter basis, 
with the \Package{chngcntr} package; if you want to have only one 
counter for the whole document, check the provided \Path{style.sty} 
file.

\marginnote[1.5\parskip]{We also load \texttt{xcolor}.}

The packages \Package{inputenc}, \Package{hyphenat}, \Package{microtype} 
are loaded in the class file. \Package{babel} and \Package{biblatex} are 
already loaded, the latter being needed to display citations in the 
margins.

\section{Document Structure}

We provide optional arguments to the \Command{title} and 
\Command{author} commands so that you can insert short, plain text 
versions of this fields, which can be used, typically in the half-title 
or somewhere else in the frontmatter, through the commands 
\Command{@plaintitle} and \Command{@plainauthor}, respectively. The 
pdftitle and pdfauthor are automatically set through hyperref to the 
plain values if present, otherwise to the normal values.

The frontmatter uses a layout without margins and a plain page style 
(\ie no headers or footers). In the mainmatter the margins are wide, the 
page numbers are arabic (while in the frontmatter there are roman 
numbers, even though they are not visible) and the headings are fancy. 
In the appendix we use \Command{bookmarksetup\{startatroot\}} so that 
the bookmarks to the chapters are on their own; without this, they would 
be under the preceding part. In the backmatter the margins shrink again 
and we reset again the bookmarks root.

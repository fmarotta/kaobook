%----------------------------------------------------------------------------------------
%	PACKAGES AND OTHER DOCUMENT CONFIGURATIONS
%----------------------------------------------------------------------------------------

\documentclass[
	fontsize=10pt,
	%oneside,
	%open=any,
	%chapterprefix=true,
	%chapterentrydots=true,
	numbers=noenddot,
	%draft,
]{kaobook}

% TODO move lots of the below to the class

% Load common packages and commands
\usepackage{styles/packages}
\usepackage{styles/commands}
\usepackage{styles/environments}
%\usepackage{styles/plaintheorems}
\usepackage{styles/mdftheorems}
\usepackage{styles/headings}
\usepackage{styles/style}

% Load packages for testing
\usepackage{blindtext}
%\usepackage{showframe}
%\usepackage{showlabels}

% Add bibfile
\addbibresource{main.bib}

% Set path for images
\graphicspath{{images/}{./}}

% Nomenclature
\makenomenclature
\renewcommand{\nomname}{Notation}
\renewcommand{\nompreamble}{The next list describes several symbols that will be later used within the body of the document.}

% Index
\makeindex[columns=3, title=Alphabetical Index, intoc]

% Glossary
\makeglossaries

\begin{document}

% General structure of a book:
% 	\frontmatter
% 	Cover
% 	Half-title (r)
% 	Frontispiece (v)
% 	Title (r)
%	Information (copyright, ISBN, etc.) (v)
%	Dedication (r)
%	Notation and other conventions used (v)
%	Table of contents (r)
%	List of figures (r)
%	Preface (r)
%
%	\mainmatter
%	Chapters (1, 2, ..., n)
%
%	\appendix
%	Appendices (A, B, ..., Z)
%
%	\backmatter
%	Bibliography (r)
%	Glossary (r)
%	Index (r)
%	Empty
%
%	Cover

\titlehead{The \texttt{kaobook} class}
\subject{Use this document as a template}
\title[Example and Documentation of the {\normalfont\texttt{kaobook}} class]{Example and Documentation \\ of the {\normalfont\texttt{kaobook}} class}
\subtitle{Customise this page according to your needs}
\author[Federico Marotta]{Federico Marotta \thanks{A \LaTeX\ lover}}
\date{\today}
\publishers{An Awesome Publisher}

\frontmatter

%\includepdf{cover-front.pdf}

%----------------------------------------------------------------------------------------
%	OPENING
%----------------------------------------------------------------------------------------

\makeatletter
\extratitle{
	% In the title page, the title is vspaced by 9.5\baselineskip
	\vspace*{9\baselineskip}
	\vspace*{\parskip}
	\begin{center}
		% In the title page, \huge is set after the komafont for title
		\usekomafont{title}\huge\@title
	\end{center}
}
\makeatother

%----------------------------------------------------------------------------------------
%	FRONTISPIECE
%----------------------------------------------------------------------------------------

% Frontispiece (verso of the half-title)

%\frontispiece{}

%----------------------------------------------------------------------------------------
%	COPYRIGHT PAGE
%----------------------------------------------------------------------------------------

\makeatletter
\uppertitleback{\@titlehead}

\lowertitleback{
	\textbf{Disclaimer}\\
	You can edit this page to suit your needs. For instance, here we have a no copyright statement, a colophon and some other information. This page is based on the corresponding page of Ken Arroyo Ohori's thesis, with minimal changes.
	
	\textbf{No copyright}\\
	\cczero\ This book is released into the public domain using the CC0 code. To the extent possible under law, I waive all copyright and related or neighbouring rights to this work.
	
	To view a copy of the CC0 code, visit: \\\url{http://creativecommons.org/publicdomain/zero/1.0/}
	
	\textbf{Colophon} \\
	This document was typeset with the help of \href{https://sourceforge.net/projects/koma-script/}{\KOMAScript} and \href{ttps://www.latex-project.org/}{\LaTeX} using the \href{https://github.com/fmarotta/kaobook/}{kaobook} class.
	
	The source code of this book is available at:\\\url{https://github.com/fmarotta/kaobook/tree/master/example}
	
	(You are welcome to contribute!)
	
	\textbf{Publisher} \\
	First printed in Jan 2019 by \@publishers
}
\makeatother

%----------------------------------------------------------------------------------------
%	DEDICATION
%----------------------------------------------------------------------------------------

\dedication{
	The harmony of the world is made manifest in Form and Number, and 
	the heart and soul and all the poetry of Natural Philosophy are 
	embodied in the concept of mathematical beauty.\\
	\flushright -- D'Arcy Wentworth Thompson
}

%----------------------------------------------------------------------------------------
%	TITLE PAGE
%----------------------------------------------------------------------------------------

% KOMA title

% If you use a cover page, you may want to assign page 3 to the title, 
% so that the cover page gets 1.
\maketitle[3]

%----------------------------------------------------------------------------------------
%	GREEK ALPHABET
% 	Originally from https://gitlab.com/jim.hefferon/linear-algebra
%----------------------------------------------------------------------------------------

\thispagestyle{empty} % Suppress headers and footers on this page

\begin{center}
	\textbf{Greek letters with pronounciation} \\[1.5ex]
	\newcommand{\pronounced}[1]{\hspace*{.2em}\small\textit{#1}}
	\begin{tabular}{l l @{\hspace*{3em}} l l}
		\toprule
		Character & Name & Character & Name \\ 
		\midrule
		$\alpha$ & alpha \pronounced{AL-fuh} & $\nu$ & nu \pronounced{NEW} \\
		$\beta$ & beta \pronounced{BAY-tuh} & $\xi$, $\Xi$ & xi \pronounced{KSIGH} \\ 
		$\gamma$, $\Gamma$ & gamma \pronounced{GAM-muh} & o & omicron \pronounced{OM-uh-CRON} \\
		$\delta$, $\Delta$ & delta \pronounced{DEL-tuh} & $\pi$, $\Pi$ & pi \pronounced{PIE} \\
		$\epsilon$ & epsilon \pronounced{EP-suh-lon} & $\rho$ & rho \pronounced{ROW} \\
		$\zeta$ & zeta \pronounced{ZAY-tuh} & $\sigma$, $\Sigma$ & sigma \pronounced{SIG-muh} \\
		$\eta$ & eta \pronounced{AY-tuh} & $\tau$ & tau \pronounced{TOW (as in cow)} \\
		$\theta$, $\Theta$ & theta \pronounced{THAY-tuh} & $\upsilon$, $\Upsilon$ & upsilon \pronounced{OOP-suh-LON} \\
		$\iota$ & iota \pronounced{eye-OH-tuh} & $\phi$, $\Phi$ & phi \pronounced{FEE, or FI (as in hi)} \\
		$\kappa$ & kappa \pronounced{KAP-uh} & $\chi$ & chi \pronounced{KI (as in hi)} \\
		$\lambda$, $\Lambda$ & lambda \pronounced{LAM-duh} & $\psi$, $\Psi$ & psi \pronounced{SIGH, or PSIGH} \\
		$\mu$ & mu \pronounced{MEW} & $\omega$, $\Omega$ & omega \pronounced{oh-MAY-guh} \\
		\bottomrule
	\end{tabular} \\[1.5ex]
	Capitals shown are the ones that differ from Roman capitals.
\end{center}

%----------------------------------------------------------------------------------------
%	PREFACE
%----------------------------------------------------------------------------------------

\chapter*{Preface}
\addcontentsline{toc}{chapter}{Preface} % Add the preface to the table of contents as a chapter

I am of the opinion that every \LaTeX\xspace geek, at least once during 
his life, feels the need to create his or her own class: this is what 
happened to me and here is the result, which, however, should be seen as 
a work still in progress. Actually, this class is not completely 
original, but it is a blend of all the best ideas that I have found in a 
number of guides, tutorials, blogs and tex.stackexchange.com posts. In 
particular, the main ideas come from two sources:

\begin{itemize}
	\item \href{https://3d.bk.tudelft.nl/ken/en/}{Ken Arroyo Ohori}'s 
	\href{https://3d.bk.tudelft.nl/ken/en/nl/ken/en/2016/04/17/a-1.5-column-layout-in-latex.html}{Doctoral 
	Thesis}, which served, with the author's permission, as a backbone 
	for the implementation of this class;
	\item The 
		\href{https://github.com/Tufte-LaTeX/tufte-latex}{Tufte-Latex 
			Class}, which was a model for the style.
\end{itemize}

The first chapter of this book is introductive and covers the most 
essential features of the class. Next, there is a bunch of chapters 
devoted to all the commands and environments that you may use in writing 
a book; in particular, it will be explained how to add notes, figures 
and tables, and references. The second part deals with the page layout 
and design, as well as additional features like coloured boxes and 
theorem environments.

I started writing this class as an experiment, and as such it should be 
regarded. Since it has always been indended for my personal use, it may 
not be perfect but I find it quite satisfactory for the use I want to 
make of it. I share this work in the hope that someone might find here 
the inspiration for writing his or her own class.

\begin{flushright}
	\textit{Federico Marotta}
\end{flushright}


%----------------------------------------------------------------------------------------
%	TABLE OF CONTENTS
%----------------------------------------------------------------------------------------

\begingroup
\etocstandarddisplaystyle
\etocstandardlines
%\etocmulticolstyle[2]{\chapter*{Contents}}
%\setstretch{1}
%\hypersetup{linkcolor=DarkBlue}
\tableofcontents
\listoffigures
\listoftables
%\listoftheorems
\endgroup

%----------------------------------------------------------------------------------------
%	MAIN BODY
%----------------------------------------------------------------------------------------

\mainmatter

\renewcommand*{\chapterformat}
{
  \enskip\mbox{\scalebox{4}{\thechapter\autodot}}
}
\renewcommand\chapterlinesformat[3]
{
  \vspace*{-1cm}%
  \parbox[b]{\textwidth}{\hrulefill#2}\par%
  #3%\par\bigskip
  \parbox[b]{\textwidth+\marginparsep+\marginparwidth}{\hrulefill}%
  %\hrule
}
\setchapterpreamble[u]{\margintoc}
\chapter{Introduction}

\section{The main ideas}

Many modern printed textbooks have adopted a layout with prominent 
margins where small figures, tables, remarks and just about everything 
else can be displayed. Arguably, this layout helps to organise the 
discussion by separating the main text from the ancillary material, 
which at the same time is very close to the point in the text where it 
is referenced.

This text does not aim to be an apology of wide margins, for there are 
many better suited authors for this task; the purpose of all these words 
is just to fill the space so that the reader can see how a book written 
with the kaobook class looks like. Meanwhile, I shall also try to 
illustrate the features of the class.

The main ideas behind kaobook come from this 
\href{https://3d.bk.tudelft.nl/ken/en/2016/04/17/a-1.5-column-layout-in-latex.html}{blog 
	post}, and actually the name of the class is dedicated to the author 
of the post, Ken Arroyo Ohori, which has kindly allowed me to create a 
class based on his thesis. Therefore, if you want to know more reasons 
to prefer a 1.5-column layout for your books, you can read his blog 
post.

\section{What this class does}
\labsec{does}

The kaobook class focuses more about the document structure than about 
the style. Indeed, it is a well-known \LaTeX\xspace printiple that 
structure and style should be separated as much as possible (see also 
\refsec{doesnot}). This means that this class will only provide 
commands, environments and in general, the opportunity to do things, 
which the user may or may not exploit. Actually, some stylistic matters 
are embedded in the class, but the user is able to customise them with 
ease.

The main features are the following:

\begin{description}
	\item[Page Headings] They span the margins and, in twoside mode, 
		display alternatively the chapter and the section name.
	\item[Matters] The commands \verb|\frontmatter|, \verb|\mainmatter| 
		and \verb|\backmatter| have been redefined in order to have 
		automatically wide margins in the main matter, and narrow 
		margins in the front and back matters.
	\item[Margin text] We provide commands \verb|\sidenote| and 
		\verb|\marginnote| to put text in the margins\sidenote{Sidenotes 
			(like this!) are numbered while marginnotes are not}.
	\item[Margin figs/tabs] A couple of useful environments is 
		\verb|marginfigure| and \verb|margintable|, which, not 
		surprisingly, allow you to put figures and tables in the margins 
		(cfr. \reffig{marginmonalisa}).
		\begin{marginfigure}[*-3]
			\includegraphics{monalisa}
			\caption[The Mona Lisa]{The Mona Lisa.\\
				\url{https://commons.wikimedia.org/wiki/File:Mona_Lisa,_by_Leonardo_da_Vinci,_from_C2RMF_retouched.jpg}}
			\labfig{marginmonalisa}
		\end{marginfigure}
	\item[Margin toc] Finally, since we have wide margins, why don't add 
		a little table of contents in them? See \verb|\margintoc| for 
		that.
	\item[Hyperref] \verb|hyperref| is loaded and by default we try to 
		add bookmarks in a sensible way; in particular, the bookmarks 
		levels are automatically reset at \verb|\appendix| and 
		\verb|\backmatter|.
\end{description}

\section{What this class does not}
\labsec{doesnot}

As anticipated, the styling is left to the user. Indeed, every book may 
have sidenotes, margin figures and so on, but each book will have its 
own fonts, toc style and so on. For this reason, we only provide 
sensible defaults. The github repository is organised as follows.

\begin{description}
	\item[kaobook.cls] The class file, which contains the definitions of 
		the commands and the environments and loads the required 
		packages.
	\item[packages.sty] Loads other packages to improve the experience 
		of the user (for instance, \verb|ams*| packages are loaded here 
		as they are not required in every book).
	\item[commands.sty] Complements to the packages, \eg the 
		specifications of the \verb|theorem| environments.
	\item[style.sty] Page layout, formatting of the titles\ldots
\end{description}

Moreover, there is a folder containing this very book as an example.


\widepage
\setpartpreamble{
	\begin{center}
		\includegraphics[width=0.8\linewidth]{images/partpre}
		\captionof{figure}[A puzzle]{\url{https://commons.wikimedia.org/wiki/File:Puzzle-4.svg}}
	\end{center}
}
\RedeclareSectionCommand[beforeskip=3cm]{part}
\addpart{Class Options, Commands and Environments}
\marginpage

\setchapterpreamble[u]{\margintoc}
\chapter{Class Options}
\labch{options}

In this chapter I will describe the most common options used, both the 
ones inherited from scrbook and the kao-specific ones.

\section{KOMA options}

The class is based on the scrbook, therefore it understands all of the 
options you would normally pass to that class. By default, the font size 
is 9.5pt and the paragraphs are separated by space, not marked by 
indentation. The default value for parskip is half.

The toc has an entry for everything: listoffigures, listoftables, 
indices, glossaries and bibliographies. There are also entries for the 
table of contents itself (thanks to the 
\Command{setuptoc\{toc\}\{totoc\}} command). If you want entries for the 
glossaries as well, you can set the \Option{toc} option of the package 
\Package{glossaries}.\sidenote[-7mm][]{If you don't want all these 
things in the table of contents, pass the appropriate KOMA options to 
the class.}

\section{KAO options}

In the future I plan to add more options to set the paragraph formatting 
(\eg justified vs ragged) and the position of the margins (inner or 
outer in twoside mode, left or right in oneside 
mode)\sidenote[-10mm][]{As of now, paragraphs are justified, formatted 
with \Command{singlespacing} (from the \Package{setspace} package) and 
\Command{frenchspacing}.}. 

\section{Other things worth knowing}

By default, dispositions are numbered up to the section thanks to the 
command \Command{setcounter\{secnumdepth\}\{1\}}. The table of contents 
can be modified through the package \Package{etoc}, which is loaded 
because it is needed for the margintocs, or the more traditional 
\Package{tocbase}. The sidenotes are numbered on a per-chapter basis, 
with the \Package{chngcntr} package; if you want to have only one 
counter for the whole document, check the provided \Path{style.sty} 
file.

\marginnote[1.5\parskip]{We also load \texttt{xcolor}.}

The packages \Package{inputenc}, \Package{hyphenat}, \Package{microtype} 
are loaded in the class file. \Package{babel} and \Package{biblatex} are 
already loaded, the latter being needed to display citations in the 
margins.

\section{Document Structure}

We provide optional arguments to the \Command{title} and 
\Command{author} commands so that you can insert short, plain text 
versions of this fields, which can be used, typically in the half-title 
or somewhere else in the frontmatter, through the commands 
\Command{@plaintitle} and \Command{@plainauthor}, respectively. The 
pdftitle and pdfauthor are automatically set through hyperref to the 
plain values if present, otherwise to the normal values.

The frontmatter uses a layout without margins and a plain page style 
(\ie no headers or footers). In the mainmatter the margins are wide, the 
page numbers are arabic (while in the frontmatter there are roman 
numbers, even though they are not visible) and the headings are fancy. 
In the appendix we use \Command{bookmarksetup\{startatroot\}} so that 
the bookmarks to the chapters are on their own; without this, they would 
be under the preceding part. In the backmatter the margins shrink again 
and we reset again the bookmarks root.

\setchapterpreamble[u]{\margintoc}
\chapter{Margin stuff}

Sidenotes are a distinctive feature of all 1.5-column-layout books. 
Indeed, having wide margins means that some material can be displayed 
there. We use margins for all kind of stuff: sidenotes, marginnotes, 
small tables of contents, citations, and, why not?, special boxes and 
environments.

\section{Sidenotes}

Sidenotes are like footnotes, except that they go in the margin, where 
they are more readable. To insert a sidenote, just use the command 
\Command{sidenote\{Text of the note\}}. You can specify a 
mark\sidenote[O]{This sidenote has a special mark, a big O!} with \\ 
\Command{sidenote[mark]\{Text\}}, but you can also specify an offset, 
which moves the sidenote upwards or downwards, so that the full syntax is:

\begin{lstlisting}[style=kaolstplain]
\sidenote[offset][mark]{Text}
\end{lstlisting}

If you use an offset, you always have to add the 
brackets for the mark, but they can be empty.\sidenote{If you want to 
know more about the usage of the \Command{sidenote} command, read the 
documentation of the \Package{snotez} package.} The format of the actual 
sidenote can be changed with the command \Command{setsidenotes}, which 
allows you to modify, for instance, the format of the markers and the 
separator between the marker and the text of the sidenote.

There was an alternative package, \Package{sidenotes}, which we could 
have used. In the end we went for \Package{snotez} because it was the 
one used in Ken Ohori's thesis, which inspired this class. The features 
are very similar, but one additional thing offered by \Package{snotez} 
is that the offset can be specified as a multiple of 
\Command{baselineskip}. For example, if you want to enter a sidenote 
with the normal mark and move it upwards one line, type:

\begin{lstlisting}[style=kaolstplain]
\sidenote[*-1][]{Text of the sidenote.}
\end{lstlisting}

Sidenotes are handled through the \Package{snotez} package, which in 
turn relies on the \Package{marginnote} package.

\section{Marginnotes}

This command is very similar to the previous one. You can create a 
marginnote with \Command{marginnote[offset]\{Text\}}, where the offset 
argument can be left out, or it can be a multiple of 
\Command{baselineskip},\marginnote[-1cm]{While the command for margin 
notes comes from the \Package{marginnote} package, it has been redefined 
in order to change the position of the optional offset argument, which 
now precedes the text of the note, whereas in the original version it 
was at the end. We have also added the possibility to use a multiple of 
\Command{baselineskip} as offset. These things were made only to make 
everything more consistent, so that you have to remember less things!} 
\eg

\begin{lstlisting}[style=kaolstplain]
\marginnote[-12pt]{Text} or \marginnote[*-3]{Text}
\end{lstlisting}

\begin{kaobox}[frametitle=To Do]
A small thing that needs to be done is to renew the \Command{sidenote} 
command so that it takes only one optional argument, the offset. The 
special mark argument can go somewhere else. In other words, we want the 
syntax of \Command{sidenote} to resemble that of \Command{marginnote}.
\end{kaobox}

We load the packages \Package{marginnote}, \Package{marginfix} and 
\Package{placeins}. Since \Package{snotez} uses \Package{marginnote}, 
what we said for marginnotes is also valid for sidenotes. Side- and 
margin- notes are shifted slightly upwards 
(\Command{renewcommand\{\textbackslash marginnotevadjust\}\{3pt\}}) in 
order to allineate them to the bottom of the line of text where the note 
is issued.

\section{Footnotes}

Even though they are not displayed in the margin, we will discuss about 
footnotes here, since sidenotes are mainly intended to be a replacement 
of them. Footnotes force the reader to constantly move from one area of 
the page to the other. Arguably, marginnotes solve this issue, so you 
should not use footnotes. Nevertheless, for completeness, we have left 
the standard command \Command{footnote}, just in case you want to put a 
footnote once in a while.\footnote{And this is how they look like. 
Notice that in the PDF file there is a back reference to the text; 
pretty cool, uh?}

\section{Margintoc}

Since we are talking about margins, we introduce here the 
\Command{margintoc} command, which allows one to put small table of 
contents in the margin. Like other commands we have discussed, 
\Command{margintoc} accepts a parameter for the vertical offset, like 
so: \Command{margintoc[offset]}.

The command can be used in any point of the document, but we think it 
makes sense to use it just at the beginning of chapters or parts. In 
this document I make use of a \KOMAScript\xspace feature and put it in 
the chapter preamble, with the following code:

\marginnote{The font used in the margintoc is the same as the one for 
	the chapter entries in the main table of contents at the beginning 
	of the document.}

\begin{lstlisting}[style=kaolstplain]
\setchapterpreamble[u]{\margintoc}
\chapter{Chapter title}
\end{lstlisting}

Not only textual stuff can be displayed in the margin, but also figures. 
Those will be the focus of the next chapter.

\setchapterimage[6cm]{seaside}
\setchapterpreamble[u]{\margintoc}
\chapter{Figures and Tables}

%\footnote[0]{The credits for the image above the chapter title go to:
	%Bushra Feroz --- Own work, CC~BY-SA~4.0, 
	%https://commons.wikimedia.org/w/index.php?curid=68724647}

\section{Normal figures and tables}

Normal figures and tables can be inserted just like in any standard 
\LaTeX\xspace document. The \verb|graphicx| package is already loaded, 
and if you want you can load \verb|subfig|. The captions will be 
positioned in the margins with the help of the \verb|floatrow| package. 
The space between the figure and the text can be specified with the 
following commands:

\begin{lstlisting}[style=kaolstplain]
\renewcommand\FBaskip{4pt}
\renewcommand\FBbskip{4pt}
\end{lstlisting}

Here is a picture of Mona Lisa (\reffig{normalmonalisa}), as an example. 
The captions are formatted as the marginnotes; to change the options you 
can use \verb|\captsetup| from the \verb|caption| package. Remember that 
if you want to reference a figure, the label must come \emph{after} the 
caption!

\begin{figure}[h]
	\includegraphics[width=0.4\textwidth]{monalisa}
	\caption[Mona Lisa, again]{It's Mona Lisa again. \blindtext}
	\labfig{normalmonalisa}
\end{figure}

The tables can be inserted as easily as the figures, as exemplified in 
the following code:

\begin{lstlisting}
\begin{table}
\begin{tabular}{ c c c c  }
	\toprule
	col1 & col2 & col3 \\
	\midrule
	\multirow{3}{4em}{Multiple row} & cell2 & cell3 \\ &
	cell5 & cell6 \\ &
	cell8 & cell9 \\
	\bottomrule
\end{tabular}
\caption[A useless table]{A useless table.}
\end{table}
\end{lstlisting}

which results in the useless \reftab{useless}.

\begin{table}[h]
\begin{tabular}{ c c c c  }
	\toprule
	col1 & col2 & col3 \\
	\midrule
	\multirow{3}{4em}{Multiple row} & cell2 & cell3 \\ &
	cell5 & cell6 \\ &
	cell8 & cell9 \\
	\bottomrule
\end{tabular}
\caption[A useless table]{A useless table.}
\labtab{useless}
\end{table}

I don't have much else to say, so I will just insert some blind text. 
\blindtext

\section{Margin figures and tables}

Marginfigures can be inserted with the environment \verb|marginfigure|. 
In this case, the whole picture is confined to the margin and the 
caption is below it. \reffig{marginmonalisa} is obtained with something 
like this:

\begin{lstlisting}
\begin{marginfigure}
	\includegraphics{monalisa}
	\caption[The Mona Lisa]{The Mona Lisa.}
	\labfig{marginmonalisa}
\end{marginfigure}
\end{lstlisting}

There is also the \verb|margintable| environment, of which 
\reftab{anotheruseless} is an example.

\begin{margintable}
\raggedright
\begin{tabular}{ c c c c }
	\hline
	col1 & col2 & col3 \\
	\hline
	\multirow{3}{4em}{Multiple row} & cell2 & cell3 \\ & cell5 & cell6 
	\\ & cell8 & cell9 \\ \hline
\end{tabular}
\caption[Another useless table]{Another useless table.}
\labtab{anotheruseless}
\end{margintable}

Marginfigures and tables can be positioned with an optional offset 
command, like so:

\begin{lstlisting}
\begin{marginfigure}[offset]
	\includegraphics{images/seaside}
\end{marginfigure}
\end{lstlisting}

Offset ca be either a measure or a multiple of \verb|\baselineskip|, 
much like with \verb|\sidenote|, \verb|\marginnote| and 
\verb|\margintoc|.\todo{improve this part} If you are wondering how I 
inserted this orange bubble, have a look at the \verb|todo| package.

\section{Wide figures and tables}

With the environments \verb|figure*| and \verb|table*| you can insert 
figures which span the whole page width. The caption will be positioned 
below.

\begin{figure*}[h!]
	\includegraphics{seaside}
	\vspace*{-1.3cm}
	\caption[A wide seaside]{A wide seaside, and a wide caption.
		Credits: By Bushra Feroz - Own work, CC BY-SA 4.0, 
		https://commons.wikimedia.org/w/index.php?curid=68724647.
		\blindtext}
\end{figure*}

\section{Image before chapter}

It is relatively easy to insert a figure before the chapter title with 
the help of the \verb|\setchapterpreamble| command. The details are left 
to the reader.\sidenote{Check the source code for a hint.}

In this chapter I also have used a different chapter title style. This 
is just to demonstrate how easy it is to alter the default if you don't 
like it and if you are willing to write some commands on your own. For 
instance, you could try the following code:

\begin{lstlisting}
\renewcommand*{\chapterformat}
{
  \enskip\mbox{\scalebox{3.5}{\framebox{\thechapter\autodot}}}
}
\renewcommand\chapterlinesformat[3]
{
  \parbox[b]{\textwidth+\marginparsep+\marginparwidth}{
	\parbox[b]{\textwidth}{#3}%
	\parbox[b]{\marginparsep}{\hfill}%
	\parbox[b]{\marginparwidth}{#2}%
  }
  %\hrule
}
\end{lstlisting}

\renewcommand*{\chapterformat}
{
  \chapappifchapterprefix{\nobreakspace}\scalebox{3.5}{\thechapter\autodot}
}
\renewcommand\chapterlinesformat[3]
{
  %\vspace*{-1cm}%
  \leavevmode
  \makebox[\textwidth+\marginparsep+\marginparwidth]{%
	\makebox[\textwidth][l]{\hrulefill[1pt]#2}%\hfill%\par%\bigskip
	\makebox[\marginparsep][l]{}%
	\makebox[\marginparwidth][l]{}%
  }\\
  %\vspace{.5cm}
  %\makebox[\textwidth+\marginparsep+\marginparwidth]{%
	%\hrulefill[1pt]%
  %}\par
  \makebox[\textwidth+\marginparsep+\marginparwidth]{%
	\makebox[\textwidth][l]{#3}%
	\makebox[\marginparsep][l]{}%
	\makebox[\marginparwidth][l]{}%
  }\\
  \makebox[\textwidth+\marginparsep+\marginparwidth]{%
	\hrulefill[1.1pt]%
	%\makebox[\textwidth][l]{\hrulefill}%
	%\makebox[\marginparsep][l]{\hrulefill}%
	%\makebox[\marginparwidth][l]{\hrulefill}%
  }
}
\RedeclareSectionCommand[beforeskip=0cm]{chapter}

\setchapterpreamble[u]{\margintoc[*-5.5]}
\chapter{References}

\section{Citations}

To cite someone \sidecite{Visscher2008,James2013} is very simple: just 
use the \verb|\sidecite| command. It does not have an offset argument 
yet, but it probably will in the future. This command supports multiple 
entries, as you can see, and by default it prints the reference on the 
margin as well as adding it to the bibliography at the end of the 
document. For this setup I used biblatex but I think that workarounds 
are possible \sidecite{James2013}. Note that the citations have nothing 
to do with the text, they are completely random as they only serve the 
purpose to illustrate the feature.

\section{Glossaries and Indices}

If you load the packages \verb|glossaries| and \verb|imakeidx| you can 
add those things to your book. For instance, I previously defined some 
glossary entries and now I am going to use them, like this: 
\gls{computer}. \verb|glossaries| allows you to use acronyms as well, 
like the following: this is the full version, \acrfull{fpsLabel}, and 
this is the short one \acrshort{fpsLabel}. These entries will appear in 
the glossary in the backmatter.

To illustrate the indexing feature,\index{index} I have just called 
\verb|\index{index}|, and an entry in the index has been added. Check it 
out!

You can read the documentation of these packages if you are interested.


\widepage
\setpartpreamble{
	\begin{center}
		\includegraphics[width=0.8\linewidth]{images/partpre}
		\captionof{figure}[A puzzle]{\url{https://commons.wikimedia.org/wiki/File:Puzzle-4.svg}}
	\end{center}
}
\RedeclareSectionCommand[beforeskip=3cm]{part}
\addpart{Design and Additional Features}
\marginpage

\setchapterpreamble[u]{\margintoc[*-6]}
\setchapterstyle{plain}
\setchapterimage[6cm]{images/seaside}
\chapter{Page Layout}

\section{Headings}

\blindtext

\section{Headers \& Footers}

\blindtext

\section{Table of Contents}

Another option that is activated by default changes the style of the 
table of contents. By default, there is an entry for everything: list of 
figures, list of tables, indices, glossaries and bibliographies. There 
are also entries for the table of contents itself (thanks to the 
\Command{setuptoc\{toc\}\{totoc\}} command). If you want entries for the 
glossaries as well, you can set the \Option{toc} option of the package 
\Package{glossaries}.\sidenote[-7mm][]{If you don't want all these 
things in the table of contents, pass the appropriate KOMA options to 
the class.}

By default, dispositions are numbered up to the section thanks to the 
command \Command{setcounter\{secnumdepth\}\{1\}}. The table of contents 
can be modified through the package \Package{etoc}, which is loaded 
because it is needed for the margintocs, or the more traditional 
\Package{tocbase}. The sidenotes are numbered on a per-chapter basis, 
with the \Package{chngcntr} package; if you want to have only one 
counter for the whole document, check the provided \Path{style.sty} 
file.

\marginnote[1.5\parskip]{We also load \texttt{xcolor}.}

The sidenote counter is reset at every chapter, but you can change that 
with the \verb|\counterwithout| command.

 The space between the figure and the text can be specified with the 
following commands:

\begin{lstlisting}[style=kaolstplain]
\renewcommand\FBaskip{4pt}
\renewcommand\FBbskip{4pt}
\end{lstlisting}


%\input{chapters/tocs} %explain how to add custom tocs
%\input{chapters/spaces}
\setchapterpreamble[u]{\margintoc}
\chapter{Mathematics and Boxes}

\section{Theorems}

\blindtext

\begin{definition}
Let $(X, d)$ be a metric space. A subset $U \subset X$ is an open set 
if, for any $x \in U$ there exists $r > 0$ such that $B(x, r) \subset 
U$. We call the topology associated to d the set $\tau\textsubscript{d}$ 
of all the open subsets of $(X, d).$
\end{definition}

\begin{theorem}
A finite intersection of open sets of (X, d) is an open set of (X, d), 
i.e $\tau\textsubscript{d}$ is closed under finite intersections. Any 
union of open sets of (X, d) is an open set of (X, d).
\end{theorem}

\begin{proposition}
A finite intersection of open sets of (X, d) is an open set of (X, d), 
i.e $\tau\textsubscript{d}$ is closed under finite intersections. Any 
union of open sets of (X, d) is an open set of (X, d).
\end{proposition}

\begin{lemma}
A finite intersection of open sets of (X, d) is an open set of (X, d), 
i.e $\tau\textsubscript{d}$ is closed under finite intersections. Any 
union of open sets of (X, d) is an open set of (X, d).
\end{lemma}

\begin{corollary}
A finite intersection of open sets of (X, d) is an open set of (X, d), 
i.e $\tau\textsubscript{d}$ is closed under finite intersections. Any 
union of open sets of (X, d) is an open set of (X, d).
\end{corollary}

\begin{proof}
The proof is left to the reader as a trivial exercise.
\end{proof}

\begin{definition}
Let $(X, d)$ be a metric space. A subset $U \subset X$ is an open set 
if, for any $x \in U$ there exists $r > 0$ such that $B(x, r) \subset 
U$. We call the topology associated to d the set $\tau\textsubscript{d}$ 
of all the open subsets of $(X, d).$
\end{definition}

\begin{example}
Let $(X, d)$ be a metric space. A subset $U \subset X$ is an open set 
if, for any $x \in U$ there exists $r > 0$ such that $B(x, r) \subset 
U$. We call the topology associated to d the set $\tau\textsubscript{d}$ 
of all the open subsets of $(X, d).$
\end{example}

\begin{examples}
\begin{subexample}
Simple example
\end{subexample}
\begin{subexample}
Simple example
\end{subexample}
\end{examples}

\begin{remark}
Let $(X, d)$ be a metric space. A subset $U \subset X$ is an open set 
if, for any $x \in U$ there exists $r > 0$ such that $B(x, r) \subset 
U$. We call the topology associated to d the set $\tau\textsubscript{d}$ 
of all the open subsets of $(X, d).$
\end{remark}

\begin{remark}
Integral $\int_{a}^{b} x^2 dx$ inline
\[\int_{a}^{b} x^2 dx\]
\end{remark}

\section{Boxes}

\blindtext

\begin{kaobox}[frametitle=Title of the box]
\blindtext
\end{kaobox}

\begin{tcbtheorem}{Fermat}{fermat}
test theorem
\end{tcbtheorem}
\begin{tcbproof}{Theorem \ref{thm:fermat}}{}
Easy.
\end{tcbproof}

%\begin{definition∗}[Inhomogeneous linear]
%\blindtext
%\end{definition∗}

\section{Experiments}

\blindtext

\marginnote{
	\begin{kaobox}[frametitle=title of margin note]
		Margin note inside a kaobox.\\
		(Actually, kaobox inside a marginnote!)
	\end{kaobox}
}

\blindtext

\begin{margintable}
	\captionsetup{type=table,position=above}
	\begin{kaobox}
		\caption{caption}
		\begin{tabular}{ |c|c|c|c| } 
			\hline
			col1 & col2 & col3 \\
			\hline
			\multirow{3}{4em}{Multiple row} & cell2 & cell3 \\ 
			& cell5 & cell6 \\ 
			& cell8 & cell9 \\ 
			\hline
		\end{tabular}
	\end{kaobox}
\end{margintable}

%\input{chapters/others} %lisings, verbatim, ... (contents of 
%packages.sty)

\appendix

% Put this in \appendix?
\widepage
\addpart{Appendix}
\marginpage

\input{chapters/ants}

\backmatter
\setchapterstyle{plain}

%----------------------------------------------------------------------------------------
%	BIBLIOGRAPHY
%----------------------------------------------------------------------------------------

%\setbibpreamble{References in citation order\par\bigskip}
\printbibliography[heading=bibintoc,title=Bibliography]

%----------------------------------------------------------------------------------------
%	INDEX
%----------------------------------------------------------------------------------------

\printindex

%----------------------------------------------------------------------------------------
%	GLOSSARY
%----------------------------------------------------------------------------------------

\newglossaryentry{computer}
{
	name=computer,
	description={is a programmable machine that receives input, stores and manipulates data, and provides output in a useful format}
}

\newacronym[longplural={Frames per Second}]{fpsLabel}{FPS}{Frame per Second}

%\setglossarystyle{listhypergroup}
\setglossarystyle{listgroup}

%\printglossary[type=\acronymtype, title=Acronyms, toctitle=Acronyms]
\printglossary[title=Special Terms, toctitle=List of terms]

%----------------------------------------------------------------------------------------
%	NOMENCLATURE
%----------------------------------------------------------------------------------------

\nomenclature{$c$}{Speed of light in a vacuum inertial frame}
\nomenclature{$h$}{Planck constant}

\printnomenclature


% Back cover
%\clearpage
%\thispagestyle{empty}
%\null%
%\clearpage
%\includepdf{pages/cover-back.pdf}

%----------------------------------------------------------------------------------------

\end{document}

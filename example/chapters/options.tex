\renewcommand*{\chapterformat}
{
  \enskip\mbox{\scalebox{3.5}{\framebox{\thechapter\autodot}}}
}
\renewcommand\chapterlinesformat[3]
{
  \parbox[b]{\textwidth+\marginparsep+\marginparwidth}{
	\parbox[b]{\textwidth}{#3}%
	\parbox[b]{\marginparsep}{\hfill}%
	\parbox[b]{\marginparwidth}{#2}%
  }
  %\hrule
}
\setchapterpreamble[u]{\margintoc[*-3]}
\chapter{Class Options}

In this chapter I will describe the most common options used, both the 
ones inherited from scrbook and the kao-specific ones.

\section{KOMA options}

The class is based on the scrbook, therefore it understands all of the 
options you would normally pass to that class. By default, the font size 
is 9pt and the paragraphs are separated by space, not marked by 
indentation. The default value for parskip is half.

The toc has an entry for everything: listoffigures, listoftables, 
indices, and bibliographies. There are also entries for the 
tableofcontents itself (through the \verb|\setuptoc{toc}{totoc}| 
command). If you want entries for the glossaries as well, you can set 
the \verb|toc| option of the package \verb|glossaries|.

\section{kao options}

In the future I plan to add more options to set the paragraph formatting 
(justified vs ragged) and the position of the margins (inner vs outer in 
twoside mode, left vs right in oneside mode)\sidenote[-5.5pt][]{As of 
now, paragraphs are justified, formatted with \texttt{singlespacing} 
(from package \texttt{setspace}) and \texttt{frenchspacing}.}. 

\section{Other things worth knowing}

By default, dispositions are numbered up to the section thanks to the 
command \verb|\setcounter{secnumdepth}{1}|. We also altered slightly the 
entries of the parts in the table of contents so as to include "Part". 
The table of contents can be modified through the package \verb|etoc|, 
which is loaded because it is needed for the margintocs.

We provide optional arguments to the \verb|\title| and \verb|\author| 
commands so that you can insert short, plain text versions of this 
fields, which can be used, typically in the half-title or somewhere else 
in the frontmatter, through the commands \verb|\plaintitle| and 
\verb|\plainauthor|, respectively.

\marginnote{We also load \texttt{xcolor}.}

The packages \verb|inputenc|, \verb|hyphenat|, \verb|microtype| are 
already loaded, but you have to load babel or polyglossia and csquotes, 
if you wish.
